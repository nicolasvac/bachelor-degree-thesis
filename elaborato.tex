\documentclass[a4paper, titlepage, 12pt, openright, twoside]{book}

\usepackage[T1]{fontenc}
\usepackage[italian]{babel}
\usepackage{frontespizio}
\usepackage{booktabs}
\usepackage{graphicx}
\usepackage{amsmath}
\usepackage{stmaryrd}
\usepackage{cite}
\usepackage[utf8]{inputenc}
\usepackage{lmodern}

\begin{document}

\begin{frontespizio}
\Universita{Verona}
\Dipartimento{Informatica}
\Corso[Laurea]{Informatica}
\Titoletto{Tesi di laurea}
\Titolo{Architetture per la gestione di ordini in locali di ristorazione}
\NCandidato{Studente}
\Candidato[436988]{VACCARI NICOLAS}
\Annoaccademico{2022-2023}
\end{frontespizio}

\tableofcontents

\chapter{Introduzione}\label{chap:introduzione}

\section{Prefazione}

Questo documento si pone come obbiettivo di analizzare i seguenti aspetti del mondo della ristorazione aziendale collettiva:
\begin{itemize}
	\item quali sono le esigenze degli enti che mettono a disposizione il servizio di ristorazione
	\item quali sono le esigenze dei gestori che operano e mantengono attivo il servizio
	\item quali problemi sussistono nella realizzazione di tali esigenze e nella loro operatività
	\item quali soluzioni digitali è possibile adottare al fine di risolvere i suddetti problemi, tenendo in considerazione
		  gli aspetti di usabilità sia verso gli operatori del servizio, sia verso gli utenti consumatori finali
\end{itemize}

Il documento non ha come obbiettivo quello di illustrare le procedure burocratiche, economiche e di approvvigionamento materie prime che riguardano il mondo della ristorazione.

\section{Panoramica sul mercato della ristorazione}

A dicembre 2022 negli archivi delle Camere di Commercio italiane risultavano attive 335.817 
imprese appartenenti al codice di attività 56.0 con il quale vengono classificati i servizi di ristorazione,
un numero che non stupisce sia per un fattore culturale, sia perché i pubblici esercizi di questo tipo sono
una realtà ampiamente diffusa in ogni regione e che non ha eguali con nessun'altra tipologia di servizio rivolta alle persone presente in Italia.
\newline
Nonostante negli ultimi 4 anni il settore abbia subito delle forti perdite dovute principalmente alla pandemia COVID-19,
all'inflazione e alla diffusione dell'ideologia "Stay at home" con annesso food delivery, 
il 2022 ha dimostrato che il mercato è in fase di stabilizzazione e ripresa, con valore stimato di circa 82 miliardi,
avvicinandosi così al valore del 2019 pari a 85.5 miliardi.
\newline
Questo settore, prevalentemente composto da manodopera umana e colmo di legislazioni da rispettare,
ha da sempre lasciato poco spazio all'introduzione della tecnologia, sia per una questione pratica di sostituzione del lavoratore,
sia per una questione di convenienza economica.
\newline
Fortunatamente per il settore IT, grazie al cambiamento dello stile di vita della persona media influenzato dai fattori citati prima,
ora i ristoratori (specialmente i ristoratori collettivi) sono sempre più alla ricerca di nuove soluzioni per attrarre nuovi clienti
e generare più vendite utilizzando gli stessi locali, aprendo così nuove possibilità di sviluppo e di conseguenza nuove sfide tecnologiche da risolvere.

\section{Tipologie di ristorazione}

Il mondo della ristorazione si divide principalmente in due categorie:
\begin{itemize}
	\item \textbf{Ristorazione commerciale}: si rivolge direttamente al consumatore finale, con una clientela di tipo occasionale
											 (in quanto sono i clienti che decidono di recarsi in quella specifica ubicazione) e nella maggior parte dei casi
											 propone alimenti che sono preparati e consumati nello stesso luogo. In confronto con la ristorazione collettiva,
											 il costo medio dei pasti è maggiore, ed il numero di individui che è possibile ospitare contemporaneamente è solitamente inferiore.
											 In questa categoria rientrano:
											 \begin{itemize}
											 	\item \textbf{Ristorazione tradizionale}: la più comune e conosciuta, include trattorie, pizzerie, bistrot, ...
											 	\item \textbf{Ristorazione agrituristica}: locali interni ad aziende agricole che servono pietanze con le loro materie prime
											 	\item \textbf{Ristorazione alberghiera}: svolta all'interno di strutture alberghiere per completare l'offerta dei propri servizi
											 	\item \textbf{Ristorazione veloce}: quella in più rapida espansione, include tutte le attività 
											 										come fastfood, snackbar, tavole calde, ...
											 \end{itemize}
	\item \textbf{Ristorazione collettiva}: si occupa della preparazione e consegna di pasti su larga scala, con alimenti preparati in grandi cucine industriali dedicate e
											successivamente distribuiti a cucine più piccole collocate all'interno dei refettori che hanno il compito di "ripristinare" la pietanza
											e consegnarla al consumatore finale. I clienti di questo servizio sono tipicamente i fruitori dell'ambiente in cui si trova il locale
											(es: i dipendenti di una azienda), diventando quindi clienti abituali e la loro scelta deriva principalmente 
											da una necessità di dover consumare almeno un pasto durante il loro periodo di permanenza nell'ubicazione. 
											I costi sono contenuti, e molto spesso una parte del costo è carico dell'ente che mette a disposizione
											il servizio, ritagliandolo come un piccolo benefit. Il modus operandi di queste imprese è quello di proporre un menù fisso,
											accompagnato da uno o più operatori che gestiscono una linea self-service dove i clienti si accodano per acquistare la pietanza.
											In questa categoria rientrano:
											\begin{itemize}
												\item \textbf{Ristorazione aziendale}: la si trova all'interno di imprese di medie o grandi dimensioni, ed è dedicata alla preparazione
																						del pranzo o cena per i dipendenti
												\item \textbf{Ristorazione scolastica}: dedicata alla preparazione di pasti per gli studenti o dipendenti di scuole e università
												\item \textbf{Ristorazione socio-sanitaria}: dedicata alla preparazione di pasti all'interno di ospedali, cliniche e case di riposo
												\item \textbf{Ristorazione assistenziale}: dedicata alla preparazione di pasti da distribuire al domicilio di persone 
																							non autosufficienti
												\item \textbf{Ristorazione comunitaria}: presente all'interno di caserme, istituti religiosi e carceri penitenziari
											\end{itemize}
\end{itemize}

\section{Il peso della ristorazione collettiva}

La ristorazione collettiva nasce con l'obbiettivo di colmare l'esigenza di tutti quegli individui
che hanno la necessità di consumare almeno un pasto fuori casa, solitamente per un fattore di tempo a disposizione.
Ad oggi possiamo considerare la ristorazione collettiva non più come "un plus" disponibile a pochi, ma una necessità di una società in continua evoluzione.
Fornendo un esempio nel settore IT, possiamo notare come le 5 aziende FAANG offrano un servizio di ristorazione di alta qualità aperto 14 ore al giorno ai propri lavoratori,
e di come questo servizio venga valorizzato da loro stessi come un punto di forza fondamentale per l'azienda stessa.
\newline
Uscendo dal mondo delle migliori aziende IT, troviamo comunque sempre più casi dove gli enti puntano ad offrire come benefit un servizio mensa di media-alta qualità,
sia ai propri dipendenti che ai propri ospiti.
\newline
Ma quanto può incidere un servizio di ristorazione?
\newline
Un caso che può far riflettere è la realtà di Google, dove per convincere i propri dipendenti a tornare in ufficio e abbandonare il lavoro da casa,
offre a chiunque accetti un servizio di creazione di una dieta bilanciata da parte di un team nutrizionale specializzato, e tutti i pasti a qualsiasi ora del giorno a carico dell'azienda.
Al di fuori di questo elemento, possiamo prendere in considerazione un articolo della Cornell University, che ci fa notare i seguenti punti:
\begin{itemize}
	\item I gruppi di individui che mangiano assieme, tendono a sviluppare rapporti umani migliori, che spesso si traduce in una produttività maggiore in gruppi di grandi dimensioni
	\item I singoli individui tendono a sentirsi più importanti e valorizzati quando il proprio ente mette a disposizione un servizio di qualità, aumentandone il morale e la motivazione
	\item Nel contesto aziendale, un dipendente può rivalutare la scelta di cambiare lavoro quando l'azienda offre un pasto garantito e di qualità,
			riducendo quindi le probabilità di turnover per l'azienda
\end{itemize}

\chapter{Caso d'uso: Refettorio aziendale collettivo}\label{chap:caso}

\section{Requisiti e obbiettivi desiderati}

\section{Problematiche ed elementi da gestire}

\section{Una prima implementazione}

\section{Fase critica: gestire l'ordine dei messaggi}

\section{Conclusioni}

\chapter{Terminologie}\label{chap:terminologie}

\begin{itemize}
	\item \textbf{FAANG}: Facebook, Apple, Amazon, Netflix, Google
\end{itemize}

\chapter{Fonti e bibliografia}\label{chap:fonti}

\bibliography{bibliography}

\chapter{Dedica}\label{chap:dedica}


\end{document}
